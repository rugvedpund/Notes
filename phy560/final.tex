\hypertarget{the-averaging-problem-and-backreaction-effects-in-general-relativity-and-cosmology}{%
\section{The Averaging Problem and Backreaction Effects in General
Relativity and
Cosmology}\label{the-averaging-problem-and-backreaction-effects-in-general-relativity-and-cosmology}}

This proposal is based broadly on the excellent review of this topic by
Ellis(2005){[}1{]}.

\hypertarget{non-commutativity-of-einsteins-field-equations-and-averaging}{%
\subsection{Non-commutativity of Einstein's Field Equations and
Averaging}\label{non-commutativity-of-einsteins-field-equations-and-averaging}}

The central issue is that any conventional averaging of the geometrical
quantities does not commute with Einstein's equations at the different
scales. For example,

\textbf{EFE}: Calculating the Einstein tensor
\(G_{1ab}\equiv R_{1ab}-\frac{1}{2}R_1g_{1ab}\) from a metric tensor
\(g_{1ab}\) at a length scale \(l_1\) and determining the quantity
\(E(g_{1ab})\equiv G_{1ab} - \kappa T_{1ab}\) where \(T_{1ab}\) is the
stress-energy tensor at the same scale \(l_1\).

\textbf{Avg}: Averaging the metric tensor
\(g_{1ab} \rightarrow g_{2ab}\) to get a smoothed metric tensor
\(g_{2ab}\) and the corresponding matter tensor
\(T_{1ab} \rightarrow T_{2ab}\) at the larger length scale
\(l_2 \gg l_1\) via a suitable averaging function
\(\braket{\cdot}: X_{ab} \mapsto \braket{X_{ab}}\)

These procedures can be shown to not commute from at least two different
perspectives:

\begin{enumerate}
\def\labelenumi{\arabic{enumi}.}
\tightlist
\item
  Averaging does not commute with scalar derivatives:
  \[\partial_i\braket{g}\neq\braket{\partial_ig}\]
\item
  The averaged inverse metric \(g^{2ab}\), that depends non-linearly on
  the tensor components of \(g_{1ab}\) is not the averaged version of
  \(g^{1ab}\): \[ \braket{g^{1ab}}\neq g^{2ab}\]

  \begin{itemize}
  \tightlist
  \item
    This implies that the resulting Christoffel symbols
    \(\Gamma^{a}_{2ab}\neq\braket{\Gamma^{a}_{1ab}}\)
  \item
    And hence the Ricci components \(R_{2ab}\neq\braket{R_{1ab}}\)
  \item
    And finally, more non-linearities in calculating the Einstein tensor
    occur when calculating \(G_{2ab}=R_{2ab}-\frac{1}{2}R_2g_{2ab}\).
  \end{itemize}
\end{enumerate}

Thus, if you smooth first and calculate the Einstein equations you fail
to account for several non-linear terms in the process than if you
calculate the Einstein equations first and then smooth:
\[E\left(\braket{g_{1ab}}\right) \neq \braket{E(g_{1ab})}\]

\hypertarget{cosmological-backreaction}{%
\subsection{Cosmological Backreaction}\label{cosmological-backreaction}}

One immediate consequence of this issue is the so-called backreaction of
perturbations in cosmology. We are mostly interested in averages of
scalar quantities like the density perturbations, rate of expansion,
etc. in order to find the best-fit cosmological parameters for an
inhomogenous universe model. The averaging of Einstein's Field Equations
then leads to the presence of an extra term:
\[ G_{ab} = T_{ab} + \Lambda g_{ab} \rightarrow \braket{G_{ab}} + \delta E_{ab} = \braket{T_{ab}} + \Lambda \braket{g_{ab}}\]
where \(\delta E_{ab}\) is the backreaction term that can be looked at
as a correction to the stress-energy tensor \(\braket{T_{ab}}\) on the
right or as a geometric correct to the Einstein tensor
\(\braket{G_{ab}}\) on the left. Such terms, due to their non-linear
dependence on the perturbations, are difficult to control.

\hypertarget{example-with-gravitational-waves}{%
\subsubsection{Example with Gravitational
Waves:}\label{example-with-gravitational-waves}}

A direct example of this appears in the calculation by Isaacson {[}2{]}
for gravitational waves where a vacuum space-time with a rapidly varying
gravitational wave appears to have an effective matter content when
viewed on larger scales. For a a slowly varying background metric
superimposed with high-frequency waves
\[g_{\mu\nu} = \gamma_{\mu\nu} + \epsilon h_{\mu\nu}, \ \partial\gamma\sim\gamma/L, \ \partial h\sim h/\lambda\]
where \(\lambda\ll L\), the Ricci tensor can be expanded as
\[R_{\mu\nu}(\gamma+h) = R^{(0)}_{\mu\nu}(\gamma) + \epsilon R^{(1)}_{\mu\nu} + \epsilon^2 R^{(2)}_{\mu\nu} + \epsilon^3 R^{(3+)}_{\mu\nu}\]
Thus, if the actual space-time is empty i.e.~\(R_{\mu\nu}=0\), then the
background metric is not that of empty space \(R^{(0)}_{\mu\nu}\neq 0\),
but can be shown to satisfy
\[ R^{(0)}_{\mu\nu} - \frac{1}{2}R^{(0)}\gamma_{\mu\nu} = -8\pi T^{\text{eff}}_{\mu\nu} \\ T^{\text{eff}}_{\mu\nu} = \frac{\epsilon^2}{8\pi}\left( R^{(2)}_{\mu\nu} - \frac{1}{2}R^{(2)}\gamma_{\mu\nu} \right)\]
which is the backreaction from the small-scale structure to the
large-scale structure.

\hypertarget{proposal}{%
\subsection{Proposal}\label{proposal}}

\hypertarget{goals}{%
\subsubsection{Goals:}\label{goals}}

\begin{enumerate}
\def\labelenumi{\arabic{enumi}.}
\tightlist
\item
  Overview of current literature for the cosmological backreaction of
  small perturbations in FLRW universe. The current leading ideas seem
  to be from Buchert(2008) {[}3{]} and Zaletdinov (1997) {[}4{]}
\item
  Construct and attempt to solve for backreaction in a Toy Model of an
  empty Euclidean universe with a scalar lump at the origin
\item
  Construct and attempt to solve for backreaction in an empty FLRW
  universe with a scalar lump at the origin.
\end{enumerate}

\hypertarget{references}{%
\subsection{References}\label{references}}

{[}1{]} Ellis, G. F. R., and T. Buchert. ``The Universe Seen at
Different Scales.'' Physics Letters A 347, no. 1--3 (November 2005):
38--46. https://doi.org/10.1016/j.physleta.2005.06.087.

{[}2{]} Isaacson, Richard A. ``Gravitational Radiation in the Limit of
High Frequency. II. Nonlinear Terms and the Effective Stress Tensor.''
Physical Review 166, no. 5 (February 25, 1968): 1272--80.
https://doi.org/10.1103/PhysRev.166.1272.

{[}3{]} Buchert, Thomas. ``Dark Energy from Structure: A Status
Report.'' General Relativity and Gravitation 40, no. 2--3 (February
2008): 467--527. https://doi.org/10.1007/s10714-007-0554-8.

{[}4{]} Zalaletdinov, Roustam M. ``Averaging Problem in General
Relativity, Macroscopic Gravity and Using Einstein's Equations in
Cosmology.'' arXiv, March 6, 1997. http://arxiv.org/abs/gr-qc/9703016.

\begin{center}\rule{0.5\linewidth}{0.5pt}\end{center}
